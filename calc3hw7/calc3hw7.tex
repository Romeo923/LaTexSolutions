\documentclass{article} % \documentclass{} is the first command in any LaTeX code.  It is used to define what kind of document 

\usepackage{amsmath} % \usepackage is a command that allows you to add functionality to your LaTeX code            
\usepackage{mathtools}
\usepackage{amssymb}
\usepackage[margin=0.75in,footskip=0.25in]{geometry}

\title{Math215} % Sets article title

\author{Homework 7, Problem 3} % Sets authors name
\date{\today} 

\begin{document} % All begin commands must be paired with an end command somewhere                    
\maketitle % creates title using infromation in preamble (title, author, date)
\section*{11.3 Problem 9} % creates a section
\begin{equation*}
  f(x,t) = e^{-t}cos(\pi x)
\end{equation*}
\begin{displaymath}
  f_{x}(x,t) = e^{-t}\frac{\partial}{\partial x}(cos(\pi x)) = e^{-t} \pi (-sin(\pi x)) = -\pi e^{-t}sin(\pi x)
\end{displaymath}
\begin{displaymath}
  f_t(x,t) = cos(\pi x) \frac{\partial}{\partial x}(e^{-t}) = cos(\pi x)(-1)e^{-t} = -e^{-t}cos(\pi x)
\end{displaymath}
\\
\begin{equation}
  f_{x}(x,t) = -\pi e^{-t}sin(\pi x)
\end{equation}
\begin{equation}
  f_t(x,t) = -e^{-t}cos(\pi x)
\end{equation}
  
\end{document}
